\hypersetup{
pdfauthor={Mauro David Dobruskin},
pdfsubject={Comunicación pariodística},
pdfsubtitle={-},
pdfauthortitle={Magister },
pdfdate={20250620110550},
pdfcopyrightstatus={Copyrighted},
pdfcaptionwriter={Alberto Moyano},
pdfcontactaddress={-},
pdfcontactcity={-},
pdfcontactpostcode={-},
pdfcontactcountry={-},
pdfcontactregion={-},
pdfcontactphone={-},
pdfcontactemail={-},
pdfcontacturl={-},
pdfpublisher={Imago Mundi},
pdftype={Text},
pdfpubtype={book},
pdfisbn={-},
pdfdoi={-},
pdfsummary={Entre los años 2009 y 2015, el debate entre periodistas sobre el periodismo y el quehacer periodístico adoptó, en Argentina, un volumen e intensidad con pocos antecedentes. Mesas redondas, jornadas, nuevos programas periodísticos en canales de aire y de cable, innumerables artículos periodísticos de crítica y defensa de periodistas se desplegaron en la prensa cotidiana y en revistas políticas. Sin embargo, este debate se destacó por la edición de un gran número de libros escritos por periodistas sobre el periodismo, los periodistas o las empresas periodísticas. Este fenómeno estuvo inscripto en el conflicto entre el gobierno de Cristina Fernández de Kirchner y los principales grupos de medios, y se vio reflejado en una intensa producción y circulación de este tipo de libros de no ficción. Periodistas en debate se propone analizar los discursos que circularon entre periodistas sobre la actividad, a los efectos de identificar algunos de los argumentos que se desplegaron para describir y disputar la voz legítima dentro del campo aludido. Estrictamente, esta obra estudia el discurso metaperiodístico a fin de avanzar en el conocimiento de los mecanismos regulatorios del campo, en los aspectos de su identidad profesional y sus intereses latentes y manifiestos, como así también en las dinámicas de legitimación y reconocimiento del campo periodístico argentino contemporáneo que se configuran en la producción ensayística de los periodistas entre los años 2009 y 2015. El estilo del programa 678, que signó una época periodística, probablemente no vuelva a encarnarse luego de haber perdido, provisoriamente, la batalla cultural que se había propuesto. Sin embargo, el campo periodístico parece haber salido debilitado de la confrontación. La crítica a su interior parece haber llegado, en la Argentina, para quedarse probablemente bajo otras formas. La experiencia no es solo local. El debilitamiento del campo y su legitimidad se observa simultáneamente en los movimientos sociales de Chile, Colombia, Ecuador y Bolivia, donde las movilizaciones sociales, a diferencia de lo que ocurría hasta hace poco tiempo, ya no buscan a los medios dominantes para ser reconocidos, sino que desconfían de ellos y son repudiados.},
pdfkeywords={Campo periodístico; Metaperiodismo; Legitimación; Independencia; Libertad de expresión},
pdfcountry={Argentina},
pdfcity={Ciudad de Buenos Aires},
pdfcopyrightstatustype={Text},
pdftitle={Periodistas en Debate.   Aportes al Estudio del Campo Periodístico.  (2009 - 2015)},
pdfcreator={gbTeXpublisher},
pdfproducer={Ecosistema de LaTeX},
unicode=true,
bookmarks=true,
pdfdisplaydoctitle=true,
pdfnewwindow=true
}
